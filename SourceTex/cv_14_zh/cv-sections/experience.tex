%----------------------------------------------------------------------------------------
%	SECTION TITLE
%----------------------------------------------------------------------------------------

\cvsection{Experience}

%----------------------------------------------------------------------------------------
%	SECTION CONTENT
%----------------------------------------------------------------------------------------

\begin{cventries}

%------------------------------------------------

\cventry
{项目经理} % Job title
{软件工程课团队项目} % Organization
{北京航空航天大学} % Location
{2015.10 - 2016. 1} % Date(s)
{ % Description(s) of tasks/responsibilities
\begin{cvitems}
\item {大三上学期,我带领团队\href{http://www.cnblogs.com/buaase}{\textbf{软剑攻城队}}完成了团队项目\href{http://buaaphylab.com/}{\textbf{物理实验网站}}的开发。}
\item {该网站在上线2个月后,注册人数已经达到了900余人,网站访问量突破2.5万次。在第一轮迭代和第二轮迭代中,我们团队均取得了团队项目第一名的优异成绩。同时,我在课程中也一直保持第一名的成绩。}
\item {在这次项目的开发经历中,我初步学会了凝聚团队,增强协作性,并努力调动成员的积极性。本次团队项目带给我的不仅仅是各种工具的熟练使用、专业技能的提升,也大大锻炼了我撰写文档、设计架构的能力。}
\end{cvitems}
}

\cventry
{开发者} % Job title
{增强现实项目 Magic Cube} % Organization
{北京航空航天大学} % Location
{2014.2 - 2014.5 } % Date(s)
{ % Description(s) of tasks/responsibilities
	\begin{cvitems}
		\item {大一下学期,因为对虚拟现实有很强的兴趣,我与同学在开源项目的基础上增加部分功能,完成了一款增强现实的应用,名为 Magic Cube。}
		\item {它可以通过扫描快递等外包装盒上印刷的二维码,利用增强现实的技术显示一些可放大查看的高精度的3D模型、视频、海报等。}
		\item {它主要使用Java开发,最终在冯如杯科技竞赛中获得了三等奖。虽然我在项目中参与开发的代码不多,但在本次项目中,我学会了结对编程与坚持不懈的精神。}
	\end{cvitems}
}


%------------------------------------------------

\end{cventries}