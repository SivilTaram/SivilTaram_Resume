%----------------------------------------------------------------------------------------
%	SECTION TITLE
%----------------------------------------------------------------------------------------

\cvsection{Writing}

%----------------------------------------------------------------------------------------
%	SECTION CONTENT
%----------------------------------------------------------------------------------------

\begin{cventries}

%------------------------------------------------

\cventry
{队长 \& 编者} % Role
{操作系统实验指导书} % Title
{Git@Oschina} % Location
{2015.7 - 2016.3} % Date(s)
{ % Description(s)
\begin{cvitems}
\item {大二下学期,我有幸完成了所有的操作系统课程实验,并在课程设计答辩时表现优异,拿到满分。该实验基于MIPS架构,移植自 MIT 6.828。但它依然有许多不完善的地方:奇怪的bug、混乱的注释与不明确的指导材料。}
\item {有感于实验本身的缺陷,为了创造更好的实验环境,我和两位同伴在阅读大量参考资料并结合自身体会的基础上,花费半年写成了《小操作系统指导手册》一书。指导书约140页,使用Latex排版,现托管在\href{https://github.com/SivilTaram/BUAAOS-guide-book}{Github}上。与此同时,我们为实验中所有的关键函数添加了完整的英文注释。}
\item {在本次撰写指导书的过程中,作为队长,我为指导书的撰写制定了有条不紊的计划,从指导书的雏形到最终团队的互相审校都比较顺利。作为编者,我负责撰写了书的第三章、第四章、第六章与Git的使用部分。}
\end{cvitems}
}

%------------------------------------------------

\end{cventries}